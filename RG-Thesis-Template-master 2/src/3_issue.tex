\chapter{本研究における問題定義と仮説}
\label{issue}

本章では、序章\ref{introduction}の内容及び背景\ref{background}の記述を踏まえて、本研究で取り組む課題とその解決要件を定義する。また、自身の仮説についても記述する。

\section{問題定義}
新型コロナウイルス感染拡大時のトイレットペーパーをめぐる流言と買いだめの発生について調査した『新型コロナウイルス感染拡大と流言・トイレットペーパー買いだめ〜報道のあり方を考える〜』~\cite{現状}の内容を参考に、現状のトイレットペーパー買いだめの状況・問題点を列挙し、整理する。

\subsection{新型コロナウイルス感染拡大時のトイレットペーパー買いだめ状況}

新型コロナウイルスの感染拡大に伴い、マスクなどの衛生用品だけでなく、生活必需品の買いだめが世界各地で起きている。
日本では2020年2月にトイレットペーパーの買いだめが発生して、全国に普及した。
各地のスーパーやドラックストア、コンビニにトイレットペーパーを買い求める人が殺到して、トイレットペーパーの品切れ状態となった。
そうした状況から、1人あたりの販売数量を制限する店舗も見られた。
買いだめの状況はその後、3月に入っても収まることはなく、入荷直後に売り切れてしまう状態が続いた。
ネット上で出品されたものの中には、通常の数倍以上の値段で取引されるものもあった。
トイレットペーパーのメーカー各社は連日24時間フル操業で生産を続けていた。
また、メーカーや卸売事業者・小売事業者が協力して店頭への配送を通常の倍に増やすなど、消費者の不安を解消するための努力が続けられた。
その結果、店頭での品薄状態は、地方では3月いっぱいでほぼ解消し、東京の都心部などでは4月中旬から下旬にかけてようやく収束に向かった。

\subsubsection{解決策の効果}

流言の打ち消しの効果\\*
実際に品切れの様子がテレビ等で報道された\\*
打ち消しするツイートは効果がなかった\\*
むしろ打ち消し情報と現実との違いに戸惑う人が増えて買いだめが加速した\\*

\subsubsection{本研究における問題定義}
予言の自己成就の概念のような、トイレットペーパーが不足するという根拠のない噂が現実になってしまうことを回避する必要がある。
実際に「トイレットペーパーが不足する」という根拠のない噂から、実際に各地のコンビニやスーパーやドラッグストアにトイレットペーパーを買い求める人びとが殺到し、店頭からトイレットペーパーが消えてしまった。
そうした問題について理解するためには、現象をコンピュータ上で再現することでそれらの構造を明らかにするべきである。
また、再現した現象に対して、問題の発生を防止するための施策を講じることが求められる。

\subsubsection{問題解決のための要件}
問題解決の要件として、1つは実社会におけるトイレットペーパー買いだめ問題の様相をシミュレーション上で正確に再現することである。
実社会で発生する問題の構造を理解するために、より正確なモデルを再現することが必要となる。
2つめは、施策の効果を検証することである。
問題の発生を防止するべく、より実社会において実現性の高い施策がどの程度効果を発揮するかを検証することが必要となる。

\subsubsection{仮説}
トイレットペーパー買いだめの様相を再現するにあたり、感染症の拡大におけるウイルスの拡散の様相を再現したシミュレーションを参考にすることができると考える。
買いだめが起きる際、まず「近いうちにトイレットペーパーが不足する」という誤った情報が拡散する。その誤情報の拡散がやがてトイレットペーパーの買いだめを引き起こすことになるが、そうした誤情報の拡散は感染症拡大におけるウイルスの拡散と同相であると考えることができる。
よって、感染症拡大のシミュレーションを参考にすることで、実社会におけるトイレットペーパー買いだめの様相を正確に再現することができる。

また、再現したシミュレーションの中で、各個人が誤った噂を広めることがないように噂の拡散に対して制限を設けること、あるいは問題に関わる個人全体に対して買いだめを自粛するように制限を設けるという対策を講じる。
以上の2つの対策により、買いだめの発生を防止して社会問題となるようなパニック状態を回避することができるのではないかと推測できる。

%%% Local Variables:
%%% mode: japanese-latex
%%% TeX-master: "./thesis"
%%% End:
