\chapter{提案手法}
\label{proposed}

本章では提案手法について述べる.

(主に感染症シミュレーションが購買シミュレーションとどう呼応しているかについて)

\section{概要}

本研究では、感染症シミュレーションを参考にして購買シミュレーションを実行する。
感染症シミュレーションにおける「各エージェントの歩幅」、「エージェント同士の接触」、「感染症への感染」という定義はぞれぞれ購買シミュレーションにおいては、「各エージェントによる噂の伝播」、「買いだめの実状の目撃」、「買いだめ行為」と考えることができる。
また、感染症シミュレーションにおける人々の「免疫獲得」という状態は、購買シミュレーションにおいては「誤った情報を拡散しない状態」であり、そうした「免疫」の獲得を通して人々は正しい認識を得た状態であると見立てることができる。
これらを踏まえて、マルチエージェントシミュレーションを行い、実社会で起きたトイレットペーパー買いだめの様相をコンピュータ上で正確に再現する。
そして買いだめによる問題の発生を確認後、それらの問題の発生を防止するために「各エージェントによる噂の伝播」と「買いだめの実情の目撃」との2点に対して制限を設けることで対策を講じ、その効果を検証する。

\section{本実験で参考にする感染症シミュレーション(背景に入れるべきか?)}
本研究で参考にしている書籍『pythonによる数値計算とシミュレーション』~\cite{感染症}の中で扱われている感染症シミュレーションについて説明する。
感染症シミュレーションにあたり、まずプログラムにより仮想世界を構築し、その中でソフトウェア製のロボットであるエージェントを動かすことを行う。
シミュレーションの対象としては、2次元平面を移動する複数のエージェントを考える。
また、エージェントはそれぞれ内部状態を持っており、環境や他のエージェントと相互作用することができる。\\*

エージェントに対しては以下のような情報を与えている。

\begin{itemize}
  \item 空間内の座標
  \item 歩幅(移動速度)
  \item 状態(健康・感染・免疫獲得)
  \item 活動(移動)制限の有無(感染状態のエージェントに対して)
  \item 活動(移動)自粛の有無(すべてのエージェントに対して)
\end{itemize}

また、各エージェントの行動ルールは以下のように定められている。

\begin{itemize}
  \item 各エージェントは空間内のみを直線的に動き回り、空間の端に到達すると反射する。
  \item 「感染」状態及びそれに関するルールは以下とする。
  \begin{itemize}
    \item 「感染」状態のエージェントが「健康」のエージェントを中心とする一定範囲R(定数)に入った場合、「健康」エージェントは「感染」へと変化する。
    \item 「感染」エージェントが活動(移動)を制限された場合、活動(移動)範囲はR/4となる。
    \item 一定期間「感染」の状態が続くとそのエージェントは「免疫獲得」となり、「健康」エージェントを「感染」へと変化させる影響力を持たなくなる。
  \end{itemize}  
\end{itemize}

\section{本実験における設定}
本研究で扱う「トイレットペーパー買いだめ」のシミュレーション設計について説明する。

\subsection{各エージェントの属性}
実社会を表した空間を用意してエージェントをランダムに配置する。
その際、各エージェントには以下のような属性を与える。

\begin{itemize}
  \item 空間内の座標
  \item 噂の拡散力(他者への影響力)
  \item 状態(影響を受けやすい・買いだめ・合理的判断・思考停止)
  \item 噂の拡散に対する制限の有無(買いだめをするエージェントに対してのみ)
  \item 噂の拡散に対する自粛の有無(すべてのエージェントに対して)

\end{itemize}

\subsubsection{各エージェントの行動ルール}
各エージェントの行動のルールは以下のように定める。

\begin{itemize}
  \item 各エージェントは空間内のみを直線的に動き回り、空間の端に到達すると反射する。
  \item 状態が思考停止となったエージェントは他者に対する影響力を失い、座標が変化しないものとする。
  \item 買いだめ状態に関するルールは以下とする。
  \begin{itemize}
    \item 「影響を受けやすい」状態のエージェントが「買いだめ」のエージェントを中心とする一定範囲R(定数)に入った場合、「影響を受けやすい」エージェントは「買いだめ」を行う。
    \item 「買いだめ」エージェントが噂の拡散を制限された場合、拡散の範囲はR/4となる。
    \item 一定期間「買いだめ」の状態が続くとそのエージェントは誤情報に気付き「合理的判断」となる。
  \end{itemize}  
  \item 「買いだめ」エージェントは一定の確率で「思考停止」の状態となる。なお、「思考停止」の状態になりうるエージェントはエージェント生成時に決定している。
\end{itemize}

以上のルールのもとで、エージェントを空間内で自由に動き回らせ、「トイレットペーパーが不足する」という誤情報がどのように広がり、それに伴って買いだめ問題の発生状況を確認する。

\subsubsection{解決策の実行}
購買シミュレーションを実装して、実社会で起きるトイレットペーパー買いだめ問題の様相を再現した後、それらの問題の発生を防止するべく、解決策となる要素を入れ込み同シミュレーション上で実行する。
本研究において筆者が提案する解決策は以下の2点である。

\begin{itemize}
  \item 「買いだめ」状態のエージェントに対して、そのエージェントの噂の拡散力に対して制限を加える。
  \item すべてのエージェントに対して、噂の拡散を自粛するように制限を加える。
\end{itemize}

1つ目については、「買いだめ」状態のエージェントに対してのみ噂の拡散を行う。2つ目は、すべてのエージェントに対して噂の拡散に制限を設けることから、1つ目よりもより高い効果が得られると考える。
また、2つの対策を組み合わせて同時に実行することで、実社会でも適用可能性のある対策としてより高い効果が得られると推測する。


%%% Local Variables:
%%% mode: japanese-latex
%%% TeX-master: "../bthesis"
%%% End:
